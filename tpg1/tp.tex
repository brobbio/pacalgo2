\documentclass[10pt, a4paper]{article}
\usepackage[paper=a4paper, left=1.5cm, right=1.5cm, bottom=1.5cm, top=3.5cm]{geometry}
\usepackage[utf8]{inputenc}
\usepackage[T1]{fontenc}
\usepackage[spanish]{babel}
\usepackage{indentfirst}
\usepackage{fancyhdr}
\usepackage{latexsym}
\usepackage{lastpage}
\usepackage{framed}
\usepackage{todonotes} % para dejar notitas de to-do!
\usepackage{aed2-symb,aed2-itef,aed2-tad}
\usepackage[colorlinks=true, linkcolor=blue]{hyperref}
\usepackage{calc}

%

% ========== Para escribir pseudo ==========
\usepackage{algorithm}
\usepackage[noend]{algpseudocode}  % "noend" es para no mostrar los endfor, endif
\algrenewcommand\alglinenumber[1]{\tiny #1:}  % Para que los numeros de linea del pseudo sean pequeños
\renewcommand{\thealgorithm}{}  % Que no aparezca el numero luego de "Algorithm"
\floatname{algorithm}{ }    % Entre {  } que quiero que aparezca en vez de "Algorithm"

% traducciones
\algrenewcommand\algorithmicwhile{\textbf{mientras}}
\algrenewcommand\algorithmicdo{\textbf{hacer}}
\algrenewcommand\algorithmicreturn{\textbf{devolver}}
\algrenewcommand\algorithmicif{\textbf{si}}
\algrenewcommand\algorithmicthen{\textbf{entonces}}
\algrenewcommand\algorithmicfor{\textbf{para}}
%=========================================================


\newcommand{\f}[1]{\text{#1}}
\renewcommand{\paratodo}[2]{$\forall~#2$: #1}
\newcommand{\numeroEjercicio}[1]{\textbf{\large{Ejercicio #1:}}\\}
\newcommand{\tituloSubEjercicio}[1]{$\newline$\tadNombre{#1:}}

\sloppy

\hypersetup{%
 % Para que el PDF se abra a página completa.
 pdfstartview= {FitH \hypercalcbp{\paperheight-\topmargin-1in-\headheight}},
 pdfauthor={Cátedra de Algoritmos y Estructuras de Datos II - DC - UBA},
 pdfkeywords={Template TADs básicos},
 pdfsubject={Template TADs básicos}
}

\parskip=5pt % 10pt es el tamaño de fuente

% Pongo en 0 la distancia extra entre ítemes.
\let\olditemize\itemize
\def\itemize{\olditemize\itemsep=0pt}

% Acomodo fancyhdr.
\pagestyle{fancy}
\thispagestyle{fancy}
\addtolength{\headheight}{1pt}
\lhead{Algoritmos y Estructuras de Datos II}
\rhead{$1^{\mathrm{er}}$ cuatrimestre de 2020}
\cfoot{\thepage /\pageref{LastPage}}
\renewcommand{\footrulewidth}{0.4pt}

\author{Algoritmos y Estructuras de Datos II, DC, UBA.}
\date{}
\title{Trabajo Pr\'actico de Algoritmos y Estructuras de Datos II}

\NeedsTeXFormat{LaTeX2e}
\ProvidesPackage{caratula}[2003/4/13 v0.1 Para componer caratulas de TPs del DC]


% ----- Imprimir un mensajito al procesar un .tex que use este package -----

\typeout{Cargando package 'caratula' v0.2 (21/4/2003)}


% ----- Algunas variables --------------------------------------------------

\let\Materia\relax
\let\Submateria\relax
\let\Titulo\relax
\let\Subtitulo\relax
\let\Grupo\relax


% ----- Comandos para que el usuario defina las variables ------------------

\def\materia#1{\def\Materia{#1}}
\def\submateria#1{\def\Submateria{#1}}
\def\titulo#1{\def\Titulo{#1}}
\def\subtitulo#1{\def\Subtitulo{#1}}
\def\grupo#1{\def\Grupo{#1}}


% ----- Token list para los integrantes ------------------------------------

\newtoks\intlist\intlist={}


% ----- Comando para que el usuario agregue integrantes

\def\integrante#1#2#3{\intlist=\expandafter{\the\intlist
    \rule{0pt}{1.2em}#1&#2&\tt #3\\[0.2em]}}


% ----- Macro para generar la tabla de integrantes -------------------------

\integrante{Church, Alonso}{1/20}{alonso@iglesia.com}
\integrante{Lovelace, Ada}{10/19}{ada\_de\_los\_dientes@tatooine.com}
\integrante{Null, Linda}{100/18}{null@null.null}
\integrante{Turing, Alan}{314/16}{halting@problem.com}

\def\tablaints{
    \begin{tabular}{|l@{\hspace{4ex}}c@{\hspace{4ex}}l|}
        \hline
        \rule{0pt}{1.2em}Integrante & LU & Correo electr\'onico\\[0.2em]
        \hline
        \the\intlist
        \hline
    \end{tabular}}

% ----- Macro para generar la parte reservada para la c�tedra -------------------------

\def\tablacatedra{%
    \\
    \textbf{Reservado para la c\'atedra}\par\bigskip
    \begin{tabular}{|c|c|c|}
        \hline
        \rule{0pt}{1.2em}Instancia & Docente & Nota\\[0.2em]
        \hline
        \rule{0pt}{1.2em}Primera entrega & \phantom{mmmmmmmmmmmmmmmmmm} & \phantom{mmmmmm} \\
        \hline
        \rule{0pt}{1.2em}Segunda entrega & & \\
        \hline
    \end{tabular}}

% ----- Codigo para manejo de errores --------------------------------------

\def\se{\let\ifsetuperror\iftrue}
\def\ifsetuperror{%
    \let\ifsetuperror\iffalse
    \ifx\Materia\relax\se\errhelp={Te olvidaste de proveer una \materia{}.}\fi
    \ifx\Titulo\relax\se\errhelp={Te olvidaste de proveer un \titulo{}.}\fi
    \edef\mlist{\the\intlist}\ifx\mlist\empty\se%
    \errhelp={Tenes que proveer al menos un \integrante{nombre}{lu}{email}.}\fi
    \expandafter\ifsetuperror}


% ----- Reemplazamos el comando \maketitle de LaTeX con el nuestro ---------

\def\maketitle{%
    \ifsetuperror\errmessage{Faltan datos de la caratula! Ingresar 'h' para mas informacion.}\fi
    \thispagestyle{empty}
    \begin{center}
    \vspace*{\stretch{2}}
    \materia{Algoritmos y Estructuras de Datos II}

    {\LARGE\textbf{\Materia}}\\[1em]
    \submateria{Trabajo Pr\'actico 1}
    \ifx\Submateria\relax\else{\Large \Submateria}\\[0.5em]\fi

%\def\titulo#1{\def\Titulo{#1}}
%\def\subtitulo#1{\def\Subtitulo{#1}}
%\def\grupo#1{\def\Grupo{#1}}
    \par\vspace{\stretch{1}}
    \titulo{El diseño contraataca}
    \subtitulo{La cosa se pone compleja}
    {\large Departamento de Computaci\'on}\\[0.5em]
    {\large Facultad de Ciencias Exactas y Naturales}\\[0.5em]
    {\large Universidad de Buenos Aires}
    \par\vspace{\stretch{3}}
    {\Large \textbf{\Titulo}}\\[0.8em]
    {\Large \Subtitulo}
    \par\vspace{\stretch{3}}
    \grupo
    \ifx\Grupo\relax\else\textbf{\Grupo}\par\bigskip\fi
    \tablaints
    \vspace*{\stretch{3}}
    \medskip
    \tablacatedra
    \end{center}
    \vspace*{\stretch{3}}
    \newpage
    }

% Comandos para cositas de complejidad

\newcommand{\bigO}{\mathcal{O}} 
\newcommand{\Nat}{\mathbb{N}}
\newcommand{\R}{\mathbb{R}}
\newcommand{\Rpos}{\mathbb{R}_{>0}}
\newcommand{\eqdef}{\overset{\mathrm{def}}{=}}
\newcommand{\eqprop}{\overset{\mathrm{prop}}{=}}
%\newcommand{\ssi}{\leftrightarrow}


\begin{document}

\materia{Coso}
\titulo{asdad}
\integrante{asd}{232}{2323}
\maketitle

\section{Introducción}
Esta es la introducción en \LaTeX.

\section{Desarrollo}
\subsection{Parte 1}


\textbf{TAD} \tadNombre{Casillero} es \tadNombre{Tupla}(nat, nat)

\textbf{TAD} \tadNombre{Fantasma} es \tadNombre{Casillero}

\textbf{TAD} \tadNombre{Pared} es \tadNombre{Casillero}

\begin{tad}{\tadNombre{Dirección}}

    \tadGeneros{dirección}

    \tadGeneradores
        \tadOperacion{arriba}{}{dirección}{}
        \tadOperacion{abajo}{}{dirección}{}
        \tadOperacion{derecha}{}{dirección}{}
        \tadOperacion{izquierda}{}{dirección}{}

\textsl{}

\end{tad}

\begin{tad}{\tadNombre{Mapa}}

    \tadUsa{Tipo1, Tipo2, Nat}
    \tadExporta{observadores, operación adicional}
    \tadGeneros{mapa}

    \tadObservadores
        \tadOperacion{fantasmas}{mapa}{conj(Fantasma)}{}
        \tadOperacion{paredes}{mapa}{conj(Pared)}{}
        \tadOperacion{dimensiones}{mapa}{tupla(nat,nat)}{}
        \tadOperacion{casilleroInicial}{mapa}{Casillero}{}
        \tadOperacion{casilleroDeLlegada}{mapa}{Casillero}{}

    \tadGeneradores
        \tadOperacion{nuevoMapa}{tupla(nat, nat)\ $d$, Casillero, Casillero, conj(Fantasma)\ $fs$, conj(Pared)\ $ps$}{mapa}{$\forall (f \in fs)(\neg \exists (p \in ps)(f = p)) \rightarrow_L ( manhattan(f, inicio) \geq 3 \wedge (ps \wedge fs) \in d)$}

    \tadAxiomas[$\ldots$]
        \tadAxioma{fantasmas(nuevoMapa($dimension$, $inicio$, $fin$, $fantasmas$, $paredes$))}{$fantasmas$}
        \tadAxioma{paredes(nuevoMapa($dimension$, $inicio$, $fin$, $fantasmas$, $paredes$))}{$paredes$}
        \tadAxioma{dimensiones(nuevoMapa($dimension$, $inicio$, $fin$, $fantasmas$, $paredes$))}{$dimension$}
        \tadAxioma{casilleroInicial(nuevoMapa($dimension$, $inicio$, $fin$, $fantasmas$, $paredes$))}{$inicio$}
        \tadAxioma{casilleroDeLlegada(nuevoMapa($dimension$, $inicio$, $fin$, $fantasmas$, $paredes$))}{$casilleroDeLlegada$}

\end{tad}

\begin{tad}{\tadNombre{Jugador}}

    \tadUsa{Tipo1, Tipo2, Nat, Mapa}
    \tadExporta{observadores, mover, estaAsustado?, gano?}
    \tadGeneros{jugador}

    \tadObservadores
        \tadOperacion{posición}{jugador}{tupla(nat, nat)}{}

    \tadGeneradores
        \tadOperacion{nuevoJugador}{tupla(nat, nat)}{jugador}{}

    \tadOtrasOperaciones
        \tadOperacion{mover}{jugador $j$, mapa, dirección}{jugador}{$\neg$estaAsustado?(j) $\wedge$ $\neg$gano?(j)}
        \tadOperacion{estaAsustado?}{jugador, mapa}{bool}{}
        \tadOperacion{gano?}{jugador, mapa}{bool}{}
        \tadOperacion{excedeElMapa?}{mapa, tupla(nat, nat)}{bool}{}

    \tadAxiomas[$\ldots$]
        \tadAxioma{posición(nuevoJugador(<$x$,$y$>))}{<$x$,$y$>}
        \tadAxioma{mover(nuevoJugador(<$x$,$y$>), $mapa$), arriba)}{\IF ($y$=0?) $\oluego$ (<$x$,$y-1$> $\in$ paredes($mapa$)) THEN nuevoJugador(<$x$,$y$>) ELSE nuevoJugador(<$x$,$y-1$>) FI}
        \tadAxioma{mover(nuevoJugador(<$x$,$y$>), $mapa$), abajo)}{\IF ($y+1 < \pi_2($dimensiones($mapa$)$)) \vee$ (<$x$,$y+1$>\ $\in$ paredes($mapa$)) THEN nuevoJugador(<$x$,$y$>) ELSE nuevoJugador(<$x$,$y+1$>) FI}
        \tadAxioma{mover(nuevoJugador(<$x$,$y$>), $mapa$), derecha)}{\IF  $(x+1 < \pi_1($dimensiones($mapa$)$))\ \vee$ (<$x+1$,$y$>\ $\in$ paredes($mapa$)) THEN nuevoJugador(<$x$,$y$>) ELSE nuevoJugador(<$x+1$,$y$>) FI}
        \tadAxioma{mover(nuevoJugador(<$x$,$y$>), $mapa$), izquierda)}{\IF ($x$=0?) $\oluego$  (<$x-1$,$y$>\ $\in$ paredes($mapa$)) THEN nuevoJugador(<$x$,$y$>) ELSE nuevoJugador(<$x-1$,$y$>) FI}
        \tadAxioma{estaAsustado?(nuevoJugador($posicionDelJugador$), $mapa$))}{$\exists$ ($f$ $\in$ fantasmas($mapa$))(manhattan($f$, $posicionDelJugador$) $\leq$ 3)}
        \tadAxioma{gano?(nuevoJugador($posicionDelJugador$), $mapa$)}{casilleroDeLlegada($mapa$) = $posicionDelJugador$}

\end{tad}

\begin{tad}{\tadNombre{Partida}}

    \tadUsa{Mapa, Jugador, Bool}
    \tadExporta{observadores, operación adicional}
    \tadGeneros{partida}

    \tadObservadores
        \tadOperacion{mapa}{}{mapa}{}
        \tadOperacion{jugador}{}{jugador}{}
        \tadOperacion{finalizó?}{}{bool}{}

    \tadGeneradores
        \tadOperacion{nuevaPartida}{mapa, jugador}{partida}{}

    \tadOtrasOperaciones
        \tadOperacion{entrada}{partida, dirección}{jugador}{$\neg$finalizó?}

    \tadAxiomas[$\ldots$]
        \tadAxioma{mapa(nuevaPartida($mapa$, $jugador$))}{$mapa$}
        \tadAxioma{jugador(nuevaPartida($mapa$, $jugador$))}{$jugador$}
        \tadAxioma{finalizó?(nuevaPartida($mapa$, $jugador$))}{estaAsustado?($jugador$, $mapa$) $\vee$ gano?($jugador$,$mapa$)}
        \tadAxioma{nuevaPartida($mapa$, $jugador$)}{posición($jugador$) = casilleroInicial($mapa$)}
        \tadAxioma{entrada($partida$, $jugador$, $dirección$)}{mover($jugador$, mapa($partida$), $dirección$)}

\end{tad}

\section{Conclusiones}
Esta cátedra es la mejor. Especialmente en humildad.
\end{document}
