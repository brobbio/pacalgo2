\documentclass[10pt, a4paper]{article}
\usepackage[paper=a4paper, left=1.5cm, right=1.5cm, bottom=1.5cm, top=3.5cm]{geometry}
\usepackage[utf8]{inputenc}
\usepackage[T1]{fontenc}
\usepackage[spanish]{babel}
\usepackage{indentfirst}
\usepackage{fancyhdr}
\usepackage{latexsym}
\usepackage{lastpage}
\usepackage{framed}
\usepackage{todonotes} % para dejar notitas de to-do!
\usepackage{aed2-symb,aed2-itef,aed2-tad}
\usepackage[colorlinks=true, linkcolor=blue]{hyperref}
\usepackage{calc}

%

% ========== Para escribir pseudo ==========
\usepackage{algorithm}
\usepackage[noend]{algpseudocode}  % "noend" es para no mostrar los endfor, endif
\algrenewcommand\alglinenumber[1]{\tiny #1:}  % Para que los numeros de linea del pseudo sean pequeños
\renewcommand{\thealgorithm}{}  % Que no aparezca el numero luego de "Algorithm"
\floatname{algorithm}{ }    % Entre {  } que quiero que aparezca en vez de "Algorithm"

% traducciones
\algrenewcommand\algorithmicwhile{\textbf{mientras}}
\algrenewcommand\algorithmicdo{\textbf{hacer}}
\algrenewcommand\algorithmicreturn{\textbf{devolver}}
\algrenewcommand\algorithmicif{\textbf{si}}
\algrenewcommand\algorithmicthen{\textbf{entonces}}
\algrenewcommand\algorithmicfor{\textbf{para}}
%=========================================================


\newcommand{\f}[1]{\text{#1}}
\renewcommand{\paratodo}[2]{$\forall~#2$: #1}
\newcommand{\numeroEjercicio}[1]{\textbf{\large{Ejercicio #1:}}\\}
\newcommand{\tituloSubEjercicio}[1]{$\newline$\tadNombre{#1:}}

\sloppy

\hypersetup{%
 % Para que el PDF se abra a página completa.
 pdfstartview= {FitH \hypercalcbp{\paperheight-\topmargin-1in-\headheight}},
 pdfauthor={Cátedra de Algoritmos y Estructuras de Datos II - DC - UBA},
 pdfkeywords={Template TADs básicos},
 pdfsubject={Template TADs básicos}
}

\parskip=5pt % 10pt es el tamaño de fuente

% Pongo en 0 la distancia extra entre ítemes.
\let\olditemize\itemize
\def\itemize{\olditemize\itemsep=0pt}

% Acomodo fancyhdr.
\pagestyle{fancy}
\thispagestyle{fancy}
\addtolength{\headheight}{1pt}
\lhead{Algoritmos y Estructuras de Datos II}
\rhead{$1^{\mathrm{er}}$ cuatrimestre de 2020}
\cfoot{\thepage /\pageref{LastPage}}
\renewcommand{\footrulewidth}{0.4pt}

\author{Algoritmos y Estructuras de Datos II, DC, UBA.}
\date{}
\title{Trabajo Pr\'actico de Algoritmos y Estructuras de Datos II}

\NeedsTeXFormat{LaTeX2e}
\ProvidesPackage{caratula}[2003/4/13 v0.1 Para componer caratulas de TPs del DC]


% ----- Imprimir un mensajito al procesar un .tex que use este package -----

\typeout{Cargando package 'caratula' v0.2 (21/4/2003)}


% ----- Algunas variables --------------------------------------------------

\let\Materia\relax
\let\Submateria\relax
\let\Titulo\relax
\let\Subtitulo\relax
\let\Grupo\relax


% ----- Comandos para que el usuario defina las variables ------------------

\def\materia#1{\def\Materia{#1}}
\def\submateria#1{\def\Submateria{#1}}
\def\titulo#1{\def\Titulo{#1}}
\def\subtitulo#1{\def\Subtitulo{#1}}
\def\grupo#1{\def\Grupo{#1}}


% ----- Token list para los integrantes ------------------------------------

\newtoks\intlist\intlist={}


% ----- Comando para que el usuario agregue integrantes

\def\integrante#1#2#3{\intlist=\expandafter{\the\intlist
    \rule{0pt}{1.2em}#1&#2&\tt #3\\[0.2em]}}


% ----- Macro para generar la tabla de integrantes -------------------------

\integrante{Church, Alonso}{1/20}{alonso@iglesia.com}
\integrante{Lovelace, Ada}{10/19}{ada\_de\_los\_dientes@tatooine.com}
\integrante{Null, Linda}{100/18}{null@null.null}
\integrante{Turing, Alan}{314/16}{halting@problem.com}

\def\tablaints{
    \begin{tabular}{|l@{\hspace{4ex}}c@{\hspace{4ex}}l|}
        \hline
        \rule{0pt}{1.2em}Integrante & LU & Correo electr\'onico\\[0.2em]
        \hline
        \the\intlist
        \hline
    \end{tabular}}

% ----- Macro para generar la parte reservada para la c�tedra -------------------------

\def\tablacatedra{%
    \\
    \textbf{Reservado para la c\'atedra}\par\bigskip
    \begin{tabular}{|c|c|c|}
        \hline
        \rule{0pt}{1.2em}Instancia & Docente & Nota\\[0.2em]
        \hline
        \rule{0pt}{1.2em}Primera entrega & \phantom{mmmmmmmmmmmmmmmmmm} & \phantom{mmmmmm} \\
        \hline
        \rule{0pt}{1.2em}Segunda entrega & & \\
        \hline
    \end{tabular}}

% ----- Codigo para manejo de errores --------------------------------------

\def\se{\let\ifsetuperror\iftrue}
\def\ifsetuperror{%
    \let\ifsetuperror\iffalse
    \ifx\Materia\relax\se\errhelp={Te olvidaste de proveer una \materia{}.}\fi
    \ifx\Titulo\relax\se\errhelp={Te olvidaste de proveer un \titulo{}.}\fi
    \edef\mlist{\the\intlist}\ifx\mlist\empty\se%
    \errhelp={Tenes que proveer al menos un \integrante{nombre}{lu}{email}.}\fi
    \expandafter\ifsetuperror}


% ----- Reemplazamos el comando \maketitle de LaTeX con el nuestro ---------

\def\maketitle{%
    \ifsetuperror\errmessage{Faltan datos de la caratula! Ingresar 'h' para mas informacion.}\fi
    \thispagestyle{empty}
    \begin{center}
    \vspace*{\stretch{2}}
    \materia{Algoritmos y Estructuras de Datos II}

    {\LARGE\textbf{\Materia}}\\[1em]
    \submateria{Trabajo Pr\'actico 1}
    \ifx\Submateria\relax\else{\Large \Submateria}\\[0.5em]\fi

%\def\titulo#1{\def\Titulo{#1}}
%\def\subtitulo#1{\def\Subtitulo{#1}}
%\def\grupo#1{\def\Grupo{#1}}
    \par\vspace{\stretch{1}}
    \titulo{El diseño contraataca}
    \subtitulo{La cosa se pone compleja}
    {\large Departamento de Computaci\'on}\\[0.5em]
    {\large Facultad de Ciencias Exactas y Naturales}\\[0.5em]
    {\large Universidad de Buenos Aires}
    \par\vspace{\stretch{3}}
    {\Large \textbf{\Titulo}}\\[0.8em]
    {\Large \Subtitulo}
    \par\vspace{\stretch{3}}
    \grupo
    \ifx\Grupo\relax\else\textbf{\Grupo}\par\bigskip\fi
    \tablaints
    \vspace*{\stretch{3}}
    \medskip
    \tablacatedra
    \end{center}
    \vspace*{\stretch{3}}
    \newpage
    }

% Comandos para cositas de complejidad

\newcommand{\bigO}{\mathcal{O}} 
\newcommand{\Nat}{\mathbb{N}}
\newcommand{\R}{\mathbb{R}}
\newcommand{\Rpos}{\mathbb{R}_{>0}}
\newcommand{\eqdef}{\overset{\mathrm{def}}{=}}
\newcommand{\eqprop}{\overset{\mathrm{prop}}{=}}
%\newcommand{\ssi}{\leftrightarrow}


\begin{document}

\materia{Coso}
\titulo{asdad}
\integrante{asd}{232}{2323}
\maketitle

\section{Introducción}
Esta es la introducción en \LaTeX.

\section{Desarrollo}
\subsection{Parte 1}

\begin{tad}{\tadNombre{Casillero}}

    \tadExtiende{Tupla(nat, nat)}

    \tadGeneros{casillero}

    \tadOtrasOperaciones
    \tadOperacion{$\bullet$ + $\bullet$}{casillero, casillero}{casillero}{}
    \tadOperacion{$\bullet$ - $\bullet$ ($c_1$, $c_2$)}{casillero, casillero}{casillero}{}

    \tadAxiomas[]
    \tadAxioma{$\pi_1$($c1$ + $c2$)}{$\pi_1$($c_1$) + $\pi_1$($c_2$)}
    \tadAxioma{$\pi_2$($c1$ + $c2$)}{$\pi_2$($c_1$) + $\pi_2$($c_2$)}
    \tadAxioma{$\pi_1$($c1$ - $c2$)}{máx\{$\pi_1$($c_1$) - $\pi_1$($c_2$), 0\}}
    \tadAxioma{$\pi_2$($c1$ - $c2$)}{máx\{$\pi_2$($c_1$) - $\pi_2$($c_2$), 0\}}

\end{tad}

\newpage

\begin{tad}{\tadNombre{Mapa}} 

    \tadUsa{Nat, Casillero}
    \tadExporta{observadores, operación adicional}
    \tadGeneros{mapa}

    \tadObservadores
        \tadOperacion{fantasmas}{mapa}{conj(casillero)}{}
        \tadOperacion{paredes}{mapa}{conj(casillero)}{}
        \tadOperacion{dimensiones}{mapa}{tupla(nat,nat)}{}
        \tadOperacion{casilleroInicial}{mapa}{casillero}{}
        \tadOperacion{casilleroDeLlegada}{mapa}{casillero}{}
        \tadOperacion{aDistanciaMenosDeN}{mapa, casillero, nat}{conj(casillero)}{}

    \tadGeneradores
        \tadOperacion{nuevoMapa}{tupla(nat; nat)\ $d$, casillero\ $inicio$, casillero\  $fin$, conj(casillero)\ $fs$, conj(casillero)\ $ps$}{mapa}{$ \emptyset?(fs \cap ps) \wedge$\\
            $ \emptyset?$(aDistanciaMenosDeN($inicio$, 3) $\cap fs$) $\wedge$\\
            $(inicio \neq fin) \wedge $\\ 
            $(\forall f \in fs)(\pi_1(f) \leq \pi_1(d) \wedge \pi_2(f) \leq \pi_2(d)) \wedge $\\
            $(\forall p \in ps)(\pi_1(p) \leq \pi_1(d) \wedge \pi_2(p) \leq \pi_2(d))$}

    \tadOtrasOperaciones
      \tadOperacion{casillerosLibres}{mapa}{conj(casillero)}{}

    \tadAxiomas[]
        \tadAxioma{fantasmas(nuevoMapa($dimension$, $inicio$, $fin$, $fs$, $ps$))}{$fs$}
        \tadAxioma{paredes(nuevoMapa($dimension$, $inicio$, $fin$, $fs$, $ps$))}{$ps$}
        \tadAxioma{dimensiones(nuevoMapa($dimension$, $inicio$, $fin$, $fs$, $ps$))}{$dimension$}
        \tadAxioma{casilleroInicial(nuevoMapa($dimension$, $inicio$, $fin$, $fs$, $ps$))}{$inicio$}
        \tadAxioma{casilleroDeLlegada(nuevoMapa($dimension$, $inicio$, $fin$, $fs$, $ps$))}{$casilleroDeLlegada$}
        \tadAxioma{aDistanciaMenosDeN($m$,\ $c$,\ $n$)}{\IF $n$=0? THEN $\{c\}$ ELSE (aDistanciaMenosDeN($c$ + $\langle$1,0$\rangle$,\ $n-1$)\ $\cup$ \\
          aDistanciaMenosDeN($c$ - $\langle$1,0$\rangle$,\ $n-1$)\ $\cup$ \\
          aDistanciaMenosDeN($c$ + $\langle$0,1$\rangle$,\ $n-1$)\ $\cup$ \\
          aDistanciaMenosDeN($c$ - $\langle$0,1$\rangle$,\ $n-1$)) \\
          $\cap$ casillerosLibres($m$) FI} 
        \tadAxioma{casillerosLibres($m$)}{\{ ($c : $casillero) \\ 
          ($\pi_1$($c$) $\leq$ $\pi_1$(dimensiones($m$)) $\wedge$ \\ 
          $\pi_2$($c$) $\leq$ $\pi_2$(dimensiones($m$))) \} - (fantasmas($m$) $\cup$ paredes($m$))}

\end{tad}


\newpage
\begin{tad}{\tadNombre{Pacman}}
\tadGeneros{Pacman}
\tadUsa {mapa}
\tadIgualdadObservacional{p_1}{p_2}{Pacman}{VerMapa($p_1$)=VerMapa($p_2$)$\wedge$ Trayectoria($p_1$)=Trayectoria($p_2$)}
\tadObservadores
\tadAlinearFunciones{HistorialMovimientos}{Pacman}
\tadOperacion{VerMapa}{Pacman}{mapa}{}
\tadOperacion{Trayectoria}{Pacman}{sec(Casilla)}{}
\tadGeneradores
\tadAlinearFunciones{InicializarJuego}{mapa\text{ } m}
\tadOperacion{InicializarJuego}{mapa\text{ } m}{p}{}
\tadOperacion{Arriba}{Pacman \text{ }p}{p}{Arriba$\in$\ DireccionesPosibles(p)\ $\wedge\ \lnot$Ganó?(p) $\wedge\lnot$\ Perdió?(p)}
\tadOperacion{Abajo}{Pacman \text{ }p}{p}{Abajo$\in$\ DireccionesPosibles(p)\ $\wedge\ \lnot$Ganó?(p) $\wedge\lnot$\ Perdió?(p)}
\tadOperacion{Derecha}{Pacman \text{ }p}{p}{Derecha$\in$\ DireccionesPosibles(p)\ $\wedge\ \lnot$Ganó?(p) $\wedge\lnot$\ Perdió?(p)}
\tadOperacion{Izquierda}{Pacman \text{ }p}{p}{Izquierda$\in$\ DireccionesPosibles(p)\ $\wedge\ \lnot$Ganó?(p) $\wedge\lnot$\ Perdió?(p)}
\tadOtrasOperaciones
\tadAlinearFunciones{DireccionesPosibles}{Pacman\text{ } p}
\tadOperacion{DireccionesPosibles}{Pacman\text{ } p}{conj(Direccion)}{}
\tadOperacion{Perdió?}{Pacman\text{ } p}{Bool}{}
\tadOperacion{Ganó?}{Pacman\text{ } p}{Bool}{}
\tadOperacion{PosicionActual}{Pacman\text{ }p}{Casilla}{}
\tadAxiomas
\tadAxioma{VerMapa(InicializarJuego m)}{m}
\tadAxioma{VerMapa(Arriba(p))}{VerMapa(p)}
\tadAxioma{VerMapa(Abajo(p))}{VerMapa(p)}
\tadAxioma{VerMapa(Izquierda(p))}{VerMapa(p)}
\tadAxioma{VerMapa(Derecha(p))}{VerMapa(p)}
\tadAlinearAxiomas{PosicionActual(InicializarJuego\text{ }m)}
\tadAxioma{AdistanciaN(p,n)}{\IF n=0 THEN $\{\text{PosicionActual}(p)\}$ ELSE AdistanciaMenosDeN(p,n)-AdistanciaMenosDeN(p,n-1) FI}
\tadAxioma{Trayectoria(InicializarJuego m)}{CasillaInicial(m)$\bullet \langle\rangle$}
\tadAxioma{Trayectoria(Arriba(p))}{$\langle\pi_1$(PosicionActual(p)),suc($\pi_2$(PosicionActual(p)))$\rangle\bullet$ Trayectoria(p)}
\tadAxioma{Trayectoria(Abajo(p))}{$\langle\pi_1$(PosicionActual(p)),pred($\pi_2$(PosicionActual(p)))$\rangle\bullet$ Trayectoria(p)}
\tadAxioma{Trayectoria(Izquierda(p))}{$\langle$ pred($\pi_1$(PosicionActual(p))),$\pi_2$(PosicionActual(p))$\rangle\bullet$ Trayectoria(p)}
\tadAxioma{Trayectoria(Derecha(p))}{$\langle$ suc($\pi_1$(PosicionActual(p))),$\pi_2$(PosicionActual(p))$\rangle\bullet$ Trayectoria(p)}
\tadAxioma{Perdió?(p)}{$\exists$ (f$\in $ fantasmas(VerMapa(p)))(f$\in$ ADistanciaMenosDeN(p,3))}
\tadAxioma{Ganó?(p)}{PosicionActual(p)=casilleroFin(VerMapa(p))}
\tadAxioma{PosicionActual(p)}{prim(Trayectoria(p))}
\tadAxioma{DireccionesPosibles(p)}{(\IF $\langle$ suc($\pi_1$(PosicionActual(p))),$\pi_2$(PosicionActual(p))$\rangle\in$ casillerosLibres(VerMapa(p)) THEN$\{$Derecha$\}$ELSE$\varnothing$ FI)$\cup$(\IF $\langle$ pred($\pi_1$(PosicionActual(p))),$\pi_2$(PosicionActual(p))$\rangle\in$ casillerosLibres(VerMapa(p)) THEN $\{$Izquierda$\}$ ELSE $\varnothing$ FI)$\cup$(\IF $\langle\pi_1$(PosicionActual(p)),suc($\pi_2$(PosicionActual(p)))$\rangle\in$ casillerosLibres(VerMapa(p)) THEN $\{Arriba\}$ ELSE $\varnothing$ FI)$\cup$(\IF $\langle \pi_1$(PosicionActual(p)),pred($\pi_2$(PosicionActual(p)))$\rangle\in$ casillerosLibres(VerMapa(p)) THEN $\{Abajo\}$ ELSE $\varnothing$ FI) }
\end{tad}


\end{document}
