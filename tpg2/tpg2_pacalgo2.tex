\documentclass[10pt, a4paper]{article}
\usepackage[paper=a4paper, left=1.5cm, right=1.5cm, bottom=1.5cm, top=3.5cm]{geometry}
\usepackage[utf8]{inputenc}
\usepackage[T1]{fontenc}
\usepackage[spanish]{babel}
\usepackage{indentfirst}
\usepackage{fancyhdr}
\usepackage{latexsym}
\usepackage{lastpage}
\usepackage{framed}
\usepackage{amsmath}
\usepackage{todonotes} % para dejar notitas de to-do!
\usepackage{aed2-symb,aed2-itef,aed2-tad}
\usepackage[colorlinks=true, linkcolor=blue]{hyperref}
\usepackage{calc}
\usepackage{relsize}
\usepackage{tikz}
\usepackage{textcomp}
\usetikzlibrary{cd}
\usepackage{verbatim}
\usepackage{mathtools}

%

% ========== Para escribir pseudo ==========
\usepackage{algorithm}
\usepackage[noend]{algpseudocode}  % "noend" es para no mostrar los endfor, endif
\algrenewcommand\alglinenumber[1]{\tiny #1:}  % Para que los numeros de linea del pseudo sean pequeños
\renewcommand{\thealgorithm}{}  % Que no aparezca el numero luego de "Algorithm"
\floatname{algorithm}{ }    % Entre {  } que quiero que aparezca en vez de "Algorithm"

% traducciones
\algrenewcommand\algorithmicwhile{\textbf{mientras}}
\algrenewcommand\algorithmicdo{\textbf{hacer}}
\algrenewcommand\algorithmicreturn{\textbf{devolver}}
\algrenewcommand\algorithmicif{\textbf{si}}
\algrenewcommand\algorithmicthen{\textbf{entonces}}
\algrenewcommand\algorithmicfor{\textbf{para}}
%=========================================================


\newcommand{\f}[1]{\text{#1}}
\renewcommand{\paratodo}[2]{$\forall~#2$: #1}
\newcommand{\numeroEjercicio}[1]{\textbf{\large{Ejercicio #1:}}\\}
\newcommand{\tituloSubEjercicio}[1]{$\newline$\tadNombre{#1:}}

\sloppy

\hypersetup{%
 % Para que el PDF se abra a página completa.
 pdfstartview= {FitH \hypercalcbp{\paperheight-\topmargin-1in-\headheight}},
 pdfauthor={Cátedra de Algoritmos y Estructuras de Datos II - DC - UBA},
 pdfkeywords={Template TADs básicos},
 pdfsubject={Template TADs básicos}
}

\parskip=5pt % 10pt es el tamaño de fuente

% Pongo en 0 la distancia extra entre ítemes.
\let\olditemize\itemize
\def\itemize{\olditemize\itemsep=0pt}

% Acomodo fancyhdr.
\pagestyle{fancy}
\thispagestyle{fancy}
\addtolength{\headheight}{1pt}
\lhead{Algoritmos y Estructuras de Datos II}
\rhead{$1^{\mathrm{er}}$ cuatrimestre de 2021}
\cfoot{\thepage /\pageref{LastPage}}
\renewcommand{\footrulewidth}{0.4pt}

\author{Algoritmos y Estructuras de Datos II, DC, UBA.}
\date{}
\title{Trabajo Pr\'actico de Algoritmos y Estructuras de Datos II}

\NeedsTeXFormat{LaTeX2e}
\ProvidesPackage{caratula}[2003/4/13 v0.1 Para componer caratulas de TPs del DC]


% ----- Imprimir un mensajito al procesar un .tex que use este package -----

\typeout{Cargando package 'caratula' v0.2 (21/4/2003)}


% ----- Algunas variables --------------------------------------------------

\let\Materia\relax
\let\Submateria\relax
\let\Titulo\relax
\let\Subtitulo\relax
\let\Grupo\relax


% ----- Comandos para que el usuario defina las variables ------------------

\def\materia#1{\def\Materia{#1}}
\def\submateria#1{\def\Submateria{#1}}
\def\titulo#1{\def\Titulo{#1}}
\def\subtitulo#1{\def\Subtitulo{#1}}
\def\grupo#1{\def\Grupo{#1}}


% ----- Token list para los integrantes ------------------------------------

\newtoks\intlist\intlist={}


% ----- Comando para que el usuario agregue integrantes

\def\integrante#1#2#3{\intlist=\expandafter{\the\intlist
    \rule{0pt}{1.2em}#1&#2&\tt #3\\[0.2em]}}


% ----- Macro para generar la tabla de integrantes -------------------------

%\integrante{Church, Alonso}{1/20}{alonso@iglesia.com}
%\integrante{Lovelace, Ada}{10/19}{ada\_de\_los\_dientes@tatooine.com}
%\integrante{Null, Linda}{100/18}{null@null.null}
%\integrante{Turing, Alan}{314/16}{halting@problem.com}

\def\tablaints{
    \begin{tabular}{|l@{\hspace{4ex}}c@{\hspace{4ex}}l|}
        \hline
        \rule{0pt}{1.2em}Integrante & LU & Correo electr\'onico\\[0.2em]
        \hline
        \the\intlist
        \hline
    \end{tabular}}

% ----- Macro para generar la parte reservada para la c�tedra -------------------------

\def\tablacatedra{%
    \\
    \textbf{Reservado para la c\'atedra}\par\bigskip
    \begin{tabular}{|c|c|c|}
        \hline
        \rule{0pt}{1.2em}Instancia & Docente & Nota\\[0.2em]
        \hline
        \rule{0pt}{1.2em}Primera entrega & \phantom{mmmmmmmmmmmmmmmmmm} & \phantom{mmmmmm} \\
        \hline
        \rule{0pt}{1.2em}Segunda entrega & & \\
        \hline
    \end{tabular}}

% ----- Codigo para manejo de errores --------------------------------------

\def\se{\let\ifsetuperror\iftrue}
\def\ifsetuperror{%
    \let\ifsetuperror\iffalse
    \ifx\Materia\relax\se\errhelp={Te olvidaste de proveer una \materia{}.}\fi
    \ifx\Titulo\relax\se\errhelp={Te olvidaste de proveer un \titulo{}.}\fi
    \edef\mlist{\the\intlist}\ifx\mlist\empty\se%
    \errhelp={Tenes que proveer al menos un \integrante{nombre}{lu}{email}.}\fi
    \expandafter\ifsetuperror}


% ----- Reemplazamos el comando \maketitle de LaTeX con el nuestro ---------

\def\maketitle{%
    \ifsetuperror\errmessage{Faltan datos de la caratula! Ingresar 'h' para mas informacion.}\fi
    \thispagestyle{empty}
    \begin{center}
    \vspace*{\stretch{2}}
    \materia{Algoritmos y Estructuras de Datos II}

    {\LARGE\textbf{\Materia}}\\[1em]
    \submateria{Trabajo Pr\'actico 1}
    \ifx\Submateria\relax\else{\Large \Submateria}\\[0.5em]\fi

%\def\titulo#1{\def\Titulo{#1}}
%\def\subtitulo#1{\def\Subtitulo{#1}}
%\def\grupo#1{\def\Grupo{#1}}
    \par\vspace{\stretch{1}}
    \titulo{Pacalgo2}
    \subtitulo{Los inertes}
    {\large Departamento de Computaci\'on}\\[0.5em]
    {\large Facultad de Ciencias Exactas y Naturales}\\[0.5em]
    {\large Universidad de Buenos Aires}
    \par\vspace{\stretch{3}}
    {\Large \textbf{\Titulo}}\\[0.8em]
    {\Large \Subtitulo}
    \par\vspace{\stretch{3}}
    \grupo
    \ifx\Grupo\relax\else\textbf{\Grupo}\par\bigskip\fi
    \tablaints
    \vspace*{\stretch{3}}
    \medskip
    \tablacatedra
    \end{center}
    \vspace*{\stretch{3}}
    \newpage
    }

% Comandos para cositas de complejidad

\newcommand{\bigO}{\mathcal{O}} 
\newcommand{\Nat}{\mathbb{N}}
\newcommand{\R}{\mathbb{R}}
\newcommand{\Rpos}{\mathbb{R}_{>0}}
\newcommand{\eqdef}{\overset{\mathrm{def}}{=}}
\newcommand{\eqprop}{\overset{\mathrm{prop}}{=}}
%\newcommand{\ssi}{\leftrightarrow}


\begin{document}

\materia{Coso}
\titulo{asdad}
\integrante{Bruno Robbio}{480/09}{brobbio@hotmail.com}
\integrante{Nicolas Andres Kinaschuk}{248/15}{nicolaskinaschuk@gmail.com}
\integrante{Pedro Joel Burgos}{804/18}{facultadburgospedrojoel@hotmail.com}
\integrante{Valentina Madelaine Saravia Ruiz}{257/18}{valentina.saraviaruiz@gmail.com}
\maketitle

\section{Introducción}
\subsection{Consideraciones}

    \begin{itemize}
       \item No se puede arrancar el mapa ganando o perdiendo 
       \item El mapa puede no tener un camino ganador o perdedor
       \item El vértice del mapa es la esquina inferior izquierda
    \end{itemize}

\section{Desarrollo}
\subsection{TP 1}

\begin{tad}{\tadNombre{Casillero}}

    \tadExtiende{Tupla(int, int)}
    \tadUsa{Int}
    \tadGeneros{casillero}
    \tadExporta{casillero, +, -, aDistanciaMenosDeN}

    \tadOtrasOperaciones
    \tadOperacion{$\bullet$ + $\bullet$}{casillero, casillero}{casillero}{}
    \tadOperacion{$\bullet$ - $\bullet$}{casillero, casillero}{casillero}{}
    \tadOperacion{aDistanciaMenosDeN}{casillero, nat}{conj(casillero)}{}

    \tadAxiomas[]
    \tadAlinearAxiomas{aDistanciaMenosDeN($c$,\ $n$)}
    \tadAxioma{$\pi_1$($c1$ + $c2$)}{$\pi_1$($c_1$) + $\pi_1$($c_2$)}
    \tadAxioma{$\pi_2$($c1$ + $c2$)}{$\pi_2$($c_1$) + $\pi_2$($c_2$)}
    \tadAxioma{$\pi_1$($c1$ - $c2$)}{$\pi_1$($c_1$) - $\pi_1$($c_2$)}
    \tadAxioma{$\pi_2$($c1$ - $c2$)}{$\pi_2$($c_1$) - $\pi_2$($c_2$)}
    \tadAxioma{aDistanciaMenosDeN($c$,\ $n$)}{\IF $n$=0? THEN $\{c\}$ ELSE (aDistanciaMenosDeN($c$ + $\langle$1,0$\rangle$,\ $n-1$)\ $\cup$ \\
          aDistanciaMenosDeN($c$ - $\langle$1,0$\rangle$,\ $n-1$)\ $\cup$ \\
          aDistanciaMenosDeN($c$ + $\langle$0,1$\rangle$,\ $n-1$)\ $\cup$ \\
          aDistanciaMenosDeN($c$ - $\langle$0,1$\rangle$,\ $n-1$)\ $\cup$ \\
          aDistanciaMenosDeN($c$,\ $n-1$)) FI} 

\end{tad}

\newpage

\begin{tad}{\tadNombre{Mapa}} 

    \tadUsa{Casillero, Conjunto, Bool}
    \tadGeneros{mapa}
    \tadExporta{mapa, observadores, generadores, casillerosLibres}
    \tadIgualdadObservacional{m_1}{m_2}{mapa}{conjFantasmas($m_1$)$\igobs$conjFantasmas($m_2$)$\ \wedge$\\
        conjParedes($m_1$)$\igobs$conjParedes($m_2$)\ $\wedge$\\
        conjChocolates($m_1$)$\igobs$conjChocolates($m_2$)\ $\wedge$\\
        dimensiones($m_1$)$\igobs$dimensiones($m_2$)\ $\wedge$\\
        casilleroInicial($m_1$)$\igobs$casilleroInicial($m_2$)\ $\wedge$\\
        vértice($m_1$)$\igobs$vértice($m_2$)\ $\wedge$\\
        casilleroDeLlegada($m_1$)$\igobs$casilleroDeLlegada($m_2$)}

    \tadObservadores
        \tadOperacion{conjFantasmas}{mapa}{conj(casillero)}{}
        \tadOperacion{conjParedes}{mapa}{conj(casillero)}{}
        \tadOperacion{conjChocolates}{mapa}{conj(casillero)}{}
        \tadOperacion{dimensiones}{mapa}{tupla(nat,nat)}{}
        \tadOperacion{vértice}{mapa}{casillero}{}
        \tadOperacion{casilleroInicial}{mapa}{casillero}{}
        \tadOperacion{casilleroDeLlegada}{mapa}{casillero}{}

    \tadGeneradores
            \quad nuevoMapa : tupla(nat, nat)\ $dimensi\acute{o}n$ $\times$ casillero\ $v\acute{e}rtice$ $\times$ casillero\ $inicio$ $\times$ casillero\ $fin$ $\times$ 

            \qquad $ $ conj(casillero)\ $fantasmas$ $\times$ conj(casillero)\ $paredes$ $\times$ conj(casillero)\ $chocolates$ $\longrightarrow$ mapa
            \begin{equation*}
            \begin{rcases}
            \begin{dcases}
            \emptyset?(fantasmas \cap paredes)\ \wedge \emptyset?(chocolates \cap paredes)\wedge \\ 
            \emptyset?(fantasmas \cap chocolates)\ \wedge (inicio \neq fin)\ \wedge \\
            dentroDeLasDimensiones(dimensi\acute{o}n, v\acute{e}rtice, inicio)\ \wedge\\
            dentroDeLasDimensiones(dimensi\acute{o}n, v\acute{e}rtice, fin)\ \wedge\\
            (\forall f \in fantasmas)(dentroDeLasDimensiones(dimensi\acute{o}n, v\acute{e}rtice, f)\ \wedge \\
            \quad (f \not \in aDistanciaMenosDeN(inicio, f, 3) \vee (inicio \in chocolates)))\ \wedge \\
            (\forall c \in chocolates)(dentroDeLasDimensiones(dimensi\acute{o}n, v\acute{e}rtice, c))\ \wedge \\
            (\forall p \in paredes)(dentroDeLasDimensiones(dimensi\acute{o}n, v\acute{e}rtice, p))
            \end{dcases}
            \end{rcases}
            \end{equation*}
    \tadOtrasOperaciones
      \tadOperacion{casillerosLibres}{mapa}{conj(casillero)}{}
      \tadOperacion{dentroDeLasDimensiones}{tupla(int, int), casillero, casillero}{bool}{}
      \tadOperacion{casillerosADerecha}{casillero\ $c$, mapa\ $m$}{conj(casillero)}{dentroDeLasDimensiones(dimensión($m$), vértice($m$), $c$)}
      \tadOperacion{casillerosIzquierda}{casillero\ $c$, mapa\ $m$}{conj(casillero)}{dentroDeLasDimensiones(dimensión($m$), vértice($m$), $c$)}
      \tadOperacion{casillerosHorizontales}{casillero\ $c$, mapa\ $m$}{conj(casillero)}{dentroDeLasDimensiones(dimensión($m$), vértice($m$), $c$)}
      \tadOperacion{casillerosArriba}{casillero\ $c$, mapa\ $m$}{conj(casillero)}{dentroDeLasDimensiones(dimensión($m$), vértice($m$), $c$)}
      \newpage
      \tadOperacion{casillerosAbajo}{casillero\ $c$, mapa\ $m$}{conj(casillero)}{dentroDeLasDimensiones(dimensión($m$), vértice($m$), $c$)}
      \tadOperacion{casillerosVerticales}{casillero\ $c$, mapa\ $m$}{conj(casillero)}{dentroDeLasDimensiones(dimensión($m$), vértice($m$), c)}
      \tadOperacion{generarCasillerosHorizontales}{mapa\ $m$, conj(casillero)\ $C$}{conj(casillero)}{($\forall\ c\in C$)(dentroDeLasDimensiones(dimensión($m$), vértice($m$), $c$))}

    \tadAxiomas[]
        \tadAxioma{conjFantasmas(nuevoMapa($dimensi\acute{o}n$, $v\acute{e}rtice$, $inicio$, $fin$, $fantasmas$, $paredes$, $chocolates$))}{$fantasmas$}
        \tadAxioma{conjParedes(nuevoMapa($dimensi\acute{o}n$, $v\acute{e}rtice$, $inicio$, $fin$, $fantasmas$, $paredes$, $chocolates$))}{$paredes$}
        \tadAxioma{conjChocolates(nuevoMapa($dimensi\acute{o}n$, $v\acute{e}rtice$, $inicio$, $fin$, $fantasmas$, $paredes$, $chocolates$))}{$chocolates$}
        \tadAxioma{dimensiones(nuevoMapa($dimensi\acute{o}n$, $v\acute{e}rtice$, $inicio$, $fin$, $fantasmas$, $paredes$, $chocolates$))}{$dimensi\acute{o}n$}
        \tadAxioma{casilleroInicial(nuevoMapa($dimensi\acute{o}n$, $v\acute{e}rtice$, $inicio$, $fin$, $fantasmas$, $paredes$, $chocolates$))}{$inicio$}
        \tadAxioma{casilleroDeLlegada(nuevoMapa($dimensi\acute{o}n$, $v\acute{e}rtice$, $inicio$, $fin$, $fantasmas$, $paredes$, $chocolates$))}{$fin$}
        \tadAxioma{vértice(nuevoMapa($dimensi\acute{o}n$, $v\acute{e}rtice$, $inicio$, $fin$, $fantasmas$, $paredes$, $chocolates$))}{$v\acute{e}rtice$}
        \tadAxioma{dentroDeLasDimensiones($dimensi\acute{o}n$, $v\acute{e}rtice$, $casilla$)}{($0 \leq \pi_1(casilla) - \pi_1(v\acute{e}rtice) < \pi_1(dimensi\acute{o}n)) \wedge \\
            (0 \leq \pi_2(casilla) - \pi_2(v\acute{e}rtice) < \pi_2(dimensi\acute{o}n)$)}
        \tadAxioma{casillerosLibres($m$)}{generarCasillerosHorizontales($m$,casillerosVerticales(vértice($m$),$m$))$-$(conjFantasmas($m$) $\cup$ conjParedes($m$))}
        \tadAxioma{generarCasillerosHorizontales($m$, casilleros)}{\IF casilleros =  $\emptyset$ THEN $\emptyset$ ELSE casillerosHorizontales(dameUno(casilleros, $m$))$\cup$\\ generarCasillerosHorizontales($m$, sinUno(casilleros))FI}
        \tadAxioma{casillerosHorizontales(c,m)}{casillerosADerecha(c,m) $\cup$ casillerosAIzquierda(c, m)}
        \tadAxioma{casillerosADerecha(c,m)}{(\IF dentroDeLasDimensiones(dimensión($m$), vértice($m$), $c+\langle1,0\rangle$) THEN  casillerosADerecha(c$+\langle 1,0\rangle$, m)
        ELSE $\emptyset$ FI)$\cup\{c\}$}
        \tadAxioma{casillerosAIzquierda(c,m)}{(\IF dentroDeLasDimensiones(dimensión($m$), vértice($m$), $c-\langle1,0\rangle$) THEN  casillerosAIzquierda(c$-\langle 1,0\rangle$, m)
        ELSE $\emptyset$ FI)$\cup\{c\}$}
        \tadAxioma{casillerosVerticales(c,m)}{casillerosArriba(c,m)$\cup$casillerosAbajo(c, m)}
        \tadAxioma{casillerosArriba(c,m)}{(\IF dentroDeLasDimensiones(dimensión($m$), vértice($m$), $c+\langle 0,1\rangle$) THEN  casillerosArriba(c$+\langle 0,1\rangle$, m)
        ELSE $\emptyset$ FI)$\cup\{c\}$}
        \tadAxioma{casillerosAbajo(c,m)}{(\IF dentroDeLasDimensiones(dimensión($m$), vértice($m$), $c-\langle 0,1\rangle$) THEN  casillerosAbajo(c$-\langle 0,1\rangle$, m)
        ELSE $\emptyset$ FI)$\cup\{c\}$} 
\end{tad}


\newpage
\begin{tad}{\tadNombre{Pacalgo2}}

    \tadUsa {Mapa}
    \tadGeneros{pacalgo2}
    \tadExporta{pacalgo2, observadores, generadores}
    \tadIgualdadObservacional{p_1}{p_2}{pacalgo2}{verMapa($p_1$)$\igobs$verMapa($p_2$)$\ \wedge$\\ 
        posiciónActual($p_1$)$\igobs$posiciónActual($p_2$)$\ \wedge$ \\
        nivelDeChocolate($p_1$)$\igobs$nivelDeChocolate($p_2$)$\ \wedge$ \\
        chocolatesRestantes($p_1$)$\igobs$chocolatesRestantes($p_2$)$\ \wedge$ \\
        ganó?($p_1$)$\igobs$ ganó?($p_2$) $\wedge$ \\
        (ganó?($p_1$) $\implies_L$ puntaje($p_1$)$\igobs$puntaje($p_2$))}

\tadObservadores
    \tadAlinearFunciones{historialMovimientos}{pacalgo2}
    \tadOperacion{verMapa}{pacalgo2}{mapa}{}
    \tadOperacion{posiciónActual}{pacalgo2}{casillero}{}
    \tadOperacion{puntaje}{pacalgo2}{nat}{ganó?($p$)}
    \tadOperacion{nivelDeChocolate}{pacalgo2}{nat}{}
    \tadOperacion{chocolatesRestantes}{pacalgo2}{conj(casillero)}{}

\tadGeneradores
    \tadAlinearFunciones{inicializarJuego}{mapa\text{ } m}
    \tadOperacion{inicializarJuego}{mapa}{pacalgo2}{}
    \tadOperacion{arriba}{pacalgo2 \text{ }$p$}{pacalgo2}{(posiciónActual($p$) + $\langle 0,1\rangle$)\ $\in$\ direccionesPosibles($p$)\ $\wedge\ \lnot$ganó?($p$) $\wedge\ \lnot$perdió?($p$)}
    \tadOperacion{abajo}{pacalgo2 \text{ }$p$}{pacalgo2}{(posiciónActual($p$) - $\langle 0,1\rangle$)\ $\in$\ direccionesPosibles($p$)\ $\wedge\ \lnot$ganó?($p$) $\wedge\ \lnot$perdió?($p$)}
    \tadOperacion{derecha}{pacalgo2 \text{ }$p$}{pacalgo2}{(posiciónActual($p$) + $\langle 1,0\rangle$)\ $\in$\ direccionesPosibles($p$)\ $\wedge\ \lnot$ganó?($p$) $\wedge\ \lnot$perdió?($p$)}
    \tadOperacion{izquierda}{pacalgo2 \text{ }$p$}{pacalgo2}{(posiciónActual($p$) - $\langle 1,0\rangle$)\ $\in$\ direccionesPosibles($p$)\ $\wedge\ \lnot$ganó?($p$) $\wedge\ \lnot$perdió?($p$)}

\tadOtrasOperaciones
    \tadAlinearFunciones{direccionesPosibles}{pacalgo2}
    \tadOperacion{direccionesPosibles}{pacalgo2}{conj(casillero)}{}
    \tadOperacion{perdió?}{pacalgo2}{bool}{}
    \tadOperacion{ganó?}{pacalgo2}{bool}{}
    \tadOperacion{pasos}{pacalgo2}{nat}{}

\tadAxiomas
\tadAlinearAxiomas{chocolatesRestantes(inicializarJuego($m$))}
    \tadAxioma{verMapa(inicializarJuego($m$))}{$m$}
    \tadAxioma{verMapa(arriba($p$))}{verMapa($p$)}
    \tadAxioma{verMapa(abajo($p$))}{verMapa($p$)}
    \tadAxioma{verMapa(izquierda($p$))}{verMapa($p$)}
    \tadAxioma{verMapa(derecha($p$))}{verMapa($p$)}
    \tadAxioma{posiciónActual(inicializarJuego $m$)}{casilleroInicial($m$)}
    \tadAxioma{posiciónActual(arriba($p$))}{posiciónActual($p$) + $\langle 0,1\rangle$}
    \tadAxioma{posiciónActual(abajo($p$))}{posiciónActual($p$) - $\langle 0,1\rangle$}
    \tadAxioma{posiciónActual(izquierda($p$))}{posiciónActual($p$) - $\langle 1,0\rangle$}
    \tadAxioma{posiciónActual(derecha($p$))}{posiciónActual($p$) + $\langle 1,0\rangle$}
    \tadAxioma{pasos(inicializarJuego $m$)}{0}
    \tadAxioma{pasos(arriba($p$))}{1 + pasos($p$)}
    \tadAxioma{pasos(abajo($p$))}{1 + pasos($p$)}
    \tadAxioma{pasos(izquierda($p$))}{1 + pasos($p$)}
    \tadAxioma{pasos(derecha($p$))}{1 + pasos($p$)}
    \tadAxioma{puntaje($p$)}{pasos($p$)}
    \tadAxioma{nivelDeChocolate(inicializarJuego($m$))}{\IF casilleroInicial($m$) $\in$ conjChocolates($m$) THEN 10 ELSE 0 FI}
    \tadAxioma{nivelDeChocolate(arriba($p$))}{\IF (posiciónActual($p$) + $\langle 0,1\rangle$) $\in$ chocolatesRestantes($p$) THEN 10 ELSE {\IF 0?(nivelDeChocolate($p$)) THEN nivelDeChocolate($p$) ELSE nivelDeChocolate($p$) - 1 FI} FI}
    \tadAxioma{nivelDeChocolate(abajo($p$))}{\IF (posiciónActual($p$) - $\langle 0,1\rangle$) $\in$ chocolatesRestantes($p$) THEN 10 ELSE {\IF 0?(nivelDeChocolate($p$)) THEN nivelDeChocolate($p$) ELSE nivelDeChocolate($p$) - 1 FI} FI}
    \tadAxioma{nivelDeChocolate(izquierda($p$))}{\IF (posiciónActual($p$) - $\langle 1,0\rangle$) $\in$ chocolatesRestantes($p$) THEN 10 ELSE {\IF 0?(nivelDeChocolate($p$)) THEN nivelDeChocolate($p$) ELSE nivelDeChocolate($p$) - 1 FI} FI}
    \tadAxioma{nivelDeChocolate(derecha($p$))}{\IF (posiciónActual($p$) + $\langle 1,0\rangle$) $\in$ chocolatesRestantes($p$) THEN 10 ELSE {\IF 0?(nivelDeChocolate($p$)) THEN nivelDeChocolate($p$) ELSE nivelDeChocolate($p$) - 1 FI} FI}
    \tadAxioma{chocolatesRestantes(inicializarJuego($m$))}{\IF casilleroInicial($m$) $\in$ conjChocolates($m$) THEN conjChocolates($m$) - \{ casilleroInicial($m$) \} ELSE conjChocolates($m$) FI}
    \tadAxioma{chocolatesRestantes(arriba($p$))}{\IF (posiciónActual($p$) + $\langle 0,1\rangle$) $\in$ chocolatesRestantes($p$) THEN chocolatesRestantes($p$) - \{ (posiciónActual($p$) + $\langle 0,1\rangle$) \} ELSE chocolatesRestantes($p$) FI}
    \tadAxioma{chocolatesRestantes(abajo($p$))}{\IF (posiciónActual($p$) - $\langle 0,1\rangle$) $\in$ chocolatesRestantes($p$) THEN chocolatesRestantes($p$) - \{ (posiciónActual($p$) - $\langle 0,1\rangle$) \} ELSE chocolatesRestantes($p$) FI}
    \tadAxioma{chocolatesRestantes(izquierda($p$))}{\IF (posiciónActual($p$) - $\langle 1,0\rangle$) $\in$ chocolatesRestantes($p$) THEN chocolatesRestantes($p$) - \{ (posiciónActual($p$) - $\langle 1,0\rangle$) \} ELSE chocolatesRestantes($p$) FI}
    \tadAxioma{chocolatesRestantes(derecha($p$))}{\IF (posiciónActual($p$) + $\langle 1,0\rangle$) $\in$ chocolatesRestantes($p$) THEN chocolatesRestantes($p$) - \{ (posiciónActual($p$) + $\langle 1,0\rangle$) \} ELSE chocolatesRestantes($p$) FI}
    \tadAxioma{perdió?($p$)}{ $\neg\emptyset$?(conjFantasmas(verMapa($p$)) $\cap$ \\ aDistanciaMenosDeN(posiciónActual($p$),3))$\wedge$ 0?(nivelDeChocolate($p$)) }
    \tadAxioma{ganó?($p$)}{posiciónActual($p$)=casilleroDeLlegada(verMapa($p$))}
    \tadAxioma{direccionesPosibles($p$)}{\IF 0?(nivelDeChocolate($p$)) THEN (aDistanciaMenosDeN(posiciónActual($p$),1)$-$posiciónActual($p$))\ $\cap$ \\ casillerosLibres(verMapa($p$)) ELSE (aDistanciaMenosDeN(posiciónActual($p$),1)$-$posiciónActual($p$))\ $\cap$ \\ (casillerosLibres(verMapa($p$))\ $\cup$ conjFantasmas(verMapa($p$)))FI}

\end{tad}

\newpage

\subsection{TP 2}

TAD persona es String

\begin{tad}{\tadNombre{Ranking}}

    \tadUsa{Int}
    \tadGeneros{ranking}
    \tadExporta{ranking, generadores, contrincante, superioresInmediatos}

    \tadObservadores
    \tadAlinearFunciones{ver}{ranking}
        \tadOperacion{ver}{ranking}{dicc(persona, nat)}{}

    \tadGeneradores
    \tadAlinearFunciones{cargarPuntaje}{ranking, persona, nat}
        \tadOperacion{iniciar}{}{ranking}{}
        \tadOperacion{cargarPuntaje}{ranking, persona, nat}{ranking}{}

    \tadOtrasOperaciones
    \tadAlinearFunciones{laOtraFunciónQueHayQueHacer}{ranking\ $r$, conj(persona)\ $nombres$, persona\ $p$}
        \tadOperacion{contrincante}{ranking\ $r$, persona\ $j$}{tupla(persona, nat)}{def?($j$, ver($r$))}
        \tadOperacion{superioresInmediatos}{ranking\ $r$, persona\ $j$}{conj(persona)}{def?($j$, ver($r$))}
        \tadOperacion{elMaximoEntreMenores}{conj($nat$)\ $puntajes$, nat\ $p$}{nat}{}
        \tadOperacion{todosLosPuntajes}{ranking\ $r$, conj(persona)\ $nombres$}{conj(nat)}{$nombres\subseteq$ claves(ver($r$))}
        \tadOperacion{jugadoresConPuntajeIgual}{ranking\ $r$, conj(persona)\ $nombres$, nat}{conj(persona)}{$nombres\subseteq$ claves(ver($r$))}
        \tadOperacion{menoresQueP}{conj(nat), nat}{conj(nat)}{}
        \tadOperacion{maximoConj}{conj(nat)\ $C$}{nat}{$\lnot\emptyset$?($C$)}

    \tadAxiomas
    \tadAlinearAxiomas{jugadoresConPuntajeIgual($r$, $nombres$, $puntaje$)}
        \tadAxioma{ver(iniciar())}{vacío}
        \tadAxioma{ver(cargarPuntaje($ranking$, $nombre$, $puntaje$))}{\IF  def?($nombre$, ver($ranking$)) THEN
                {\IF $puntaje$ < obtener(ver($r$), $nombre$) THEN
                    definir(ver($ranking$), ($nombre$, $puntaje$))
                ELSE
                    ver($ranking$)
                FI}
            ELSE
                definir(ver($ranking$), ($nombre$, $puntaje$))
            FI}
        \tadAxioma{contrincante($r$,$j$)}{(dameUno(superioresInmediatos($r$,$j$)), elMaximoEntreMenores(todosLosPuntajes($r$),\\ obtener(ver($r$,$j$))))}
        \tadAxioma{superioresInmediatos($r$, $j$)}{jugadoresConPuntajeIgual($r$,claves(ver($r$)), elMaximoEntreMenores(todosLosPuntajes($r$), obtener(ver($r$),$j$)))}
        \tadAxioma{elMaximoEntreMenores($puntajes$, $p$)}{\IF $\emptyset$?(menoresQueP($puntajes$, $p$)) THEN $p$ ELSE maximoConj(menoresQueP($puntajes$, $p$)) FI}
        \tadAxioma{menoresQueP($puntajes$, $p$)}{\IF $\emptyset$?($puntajes$) THEN $\emptyset$ ELSE {\IF dameUno($puntajes$)<p THEN Ag(dameUno($puntajes$), menoresQueP(sinUno($puntajes$), $p$)) ELSE menoresQueP(sinUno($puntajes$), $p$) FI} FI}
        \tadAxioma{maximoConj($C$)}{\IF $\emptyset$?(sinUno($C$)) THEN dameUno($C$) ELSE $\max$(dameUno($C$), maximoConj(sinUno($C$))) FI}    
        \tadAxioma{todosLosPuntajes($r$)}{\IF $\emptyset$?(ver($r$)) THEN
                $\emptyset$
            ELSE
                Ag(obtener(ver($r$), dameUno(claves(ver($r$)))), todosLosPuntajes($r$))
            FI}
        \tadAxioma{jugadoresConPuntajeIgual($r$, $nombres$, $puntaje$)}{\IF $\emptyset$?($nombres$) $\vee$ $\emptyset$?(claves(ver($r$))) THEN
                $\emptyset$
            ELSE
                {\IF obtener(ver($r$), dameUno($nombres$)) = $puntaje$ THEN
                    Ag(dameUno($nombres$), jugadorConPuntajeIgual($r$, sinUno($nombres$), $puntaje$))
                ELSE
                    jugadorConPuntajeIgual($r$,sinUno($nombres$),$puntaje$)
                FI}
            FI}

\end{tad}


\begin{tad}{\tadNombre{Fichín}}

    \tadUsa{Int, Ranking, Paclgo2}
    \tadGeneros{fichin}
    \tadExporta{observadores, generadores, otras operaciones}

    \tadObservadores
        \tadOperacion{conocerRanking}{fichin}{ranking}{}
        \tadOperacion{verJuegoActual}{fichin\ $f$}{paclgo2}{$\neg$estáLibre?($f$)}
        \tadOperacion{jugadorActual}{fichin\ $f$}{persona}{$\neg$estáLibre?($f$)}

    \tadGeneradores
        \tadOperacion{nuevoFichin}{}{fichin}{}
        \tadOperacion{nuevaPartida}{fichin\ $f$, pacalgo2\ $p$, persona\ $j$}{fichin}{estáLibre?($f$)}
        \tadOperacion{palancaArriba}{fichin\ $f$}{fichin}{(posiciónActual(verJuegoActual($f$)) + $\langle 0,1\rangle$)\ $\in$\ direccionesPosibles(verJuegoActual($f$))\ $\wedge$\\
        $\neg$estáLibre?($f$)}
        \tadOperacion{palancaAbajo}{fichin\ $f$}{fichin}{(posiciónActual(verJuegoActual($f$)) + $\langle 0,-1\rangle$)\ $\in$\ direccionesPosibles(verJuegoActual($f$))\ $\wedge$\\
        $\neg$estáLibre?($f$)}
        \tadOperacion{palancaDerecha}{fichin\ $f$}{fichin}{(posiciónActual(verJuegoActual($f$)) + $\langle 1,0\rangle$)\ $\in$\ direccionesPosibles(verJuegoActual($f$))\ $\wedge$\\
        $\neg$estáLibre?($f$)}
        \tadOperacion{palancaIzquierda}{fichin\ $f$}{fichin}{(posiciónActual(verJuegoActual($f$)) + $\langle -1,0\rangle$)\ $\in$\ direccionesPosibles(verJuegoActual($f$))\ $\wedge$\\
        $\neg$estáLibre?($f$)}

    \tadOtrasOperaciones
        \tadOperacion{puntajeDeJugadorActual}{fichin\ $f$}{nat}{$\neg$estálibre?($f$)\ $\wedge_L$jugadorActual($f$)$\in$claves(ver(conocerRanking($f$)))}
        \tadOperacion{verContrincante}{fichin\ $f$}{persona}{$\neg$estáLibre?($f$)$\wedge_L$jugadorActual($f$)$\in$claves(ver(conocerRanking($f$)))}
        \tadOperacion{estáLibre?}{fichin}{Bool}{}

    \tadAxiomas
    \tadAlinearAxiomas{verJuegoActual(palancaIzquierda($f$))}
        \tadAxioma{estáLibre?(nuevoFichin())}{True}
        \tadAxioma{estáLibre?(nuevaPartida($f$,$p$,$j$))}{False}
        \tadAxioma{estáLibre?(palancaArriba($f$))}{\IF ganó?(arriba(verJuegoActual($f$)))$\vee$perdió?(arriba(verJuegoActual($f$))) THEN True ELSE False FI}
        \tadAxioma{estáLibre?(palancaAbajo($f$))}{\IF ganó?(abajo(verJuegoActual($f$)))$\vee$perdió?(abajo(verJuegoActual($f$))) THEN True ELSE False FI}
        \tadAxioma{estáLibre?(palancaDerecha($f$))}{\IF ganó?(derecha(verJuegoActual($f$)))$\vee$perdió?(derecha(verJuegoActual($f$))) THEN True ELSE False FI}
        \tadAxioma{estáLibre?(palancaIzquierda($f$))}{\IF ganó?(izquierda(verJuegoActual($f$)))$\vee$perdió?(izquierda(verJuegoActual($f$))) THEN True ELSE False FI}
        \tadAxioma{conocerRanking(nuevoFichin())}{iniciar()}
        \tadAxioma{conocerRanking(nuevaPartida($f$,$p$,$j$))}{conocerRanking($f$)}
        \tadAxioma{conocerRanking(palancaArriba($f$)}{\IF ganó?(arriba(verJuegoActual($f$))) THEN
                cargarPuntaje(conocerRanking($f$), puntaje(verJuegoActual(palancaArriba($f$))))
            ELSE 
                conocerRanking($f$)
            FI}
        \tadAxioma{conocerRanking(palancaAbajo($f$)}{\IF ganó?(abajo(verJuegoActual($f$))) THEN
                cargarPuntaje(conocerRanking($f$), puntaje(verJuegoActual(palancaAbajo($f$))))
            ELSE 
                conocerRanking($f$)
            FI}
        \tadAxioma{conocerRanking(palancaDerecha($f$)}{\IF ganó?(derecha(verJuegoActual($f$))) THEN
                cargarPuntaje(conocerRanking($f$), puntaje(verJuegoActual(palancaDerecha($f$))))
            ELSE 
                conocerRanking($f$)
            FI}
        \tadAxioma{conocerRanking(palancaIzquierda($f$)}{\IF ganó?(izquierda(verJuegoActual($f$))) THEN
                cargarPuntaje(conocerRanking($f$), puntaje(verJuegoActual(palancaIzquierda($f$))))
            ELSE 
                conocerRanking($f$)
            FI}

        \tadAxioma{verJuegoActual(nuevaPartida($f$, $p$, $j$)}{$p$}
        \tadAxioma{verJuegoActual(palancaArriba($f$))}{arriba(verJuegoActual($f$))}
        \tadAxioma{verJuegoActual(palancaAbajo($f$))}{abajo(verJuegoActual($f$))}
        \tadAxioma{verJuegoActual(palancaDerecha($f$))}{derecha(verJuegoActual($f$))}
        \tadAxioma{verJuegoActual(palancaIzquierda($f$))}{izquierda(verJuegoActual($f$))}
        \tadAxioma{verContrincante($f$)}{contrincante(conocerRanking($f$),jugadorActual($f$))}
        \tadAxioma{puntajeDeJugadorActual($f$)}{obtener(ver(conocerRanking($f$)),jugadorActual($f$))}
\end{tad}

\end{document}
